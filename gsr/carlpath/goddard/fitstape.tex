\documentstyle[12pt]{report}
\newcommand{\exbegin}{\par\medskip}
\newcommand{\exend}{\medskip\noindent}
\newcommand{\exc}[2]{
\hbox to \hsize{\small\hskip .2in
\parbox[t]{2.2in}{\raggedright\setlength{\parindent}{-.2in}\tt #1}
\hspace{.2in}
\parbox[t]{3.4in}{\raggedright\setlength{\parindent}{-.2in}\rm #2}\hss}
\prevdepth=1.5pt\relax}
% One line example
\newcommand{\exone}[1]{\begin{center}\tt #1 \end{center}}
\begin{document}
\section{IDL FITS tape I/O}
{\bf FITSRD - IDL FITS tape read routine} \\

To read a FITS tape from IDL and place the data into SDAS internal FITS
formatted data sets use
the routine FITSRD.  To execute, while in IDL, type:
\begin{center}
{\tt FITSRD}
\end{center}
You will then be asked for the tape unit number.  For example if
the tape is mounted on a drive which was assigned to MT0:,
type:
\begin{center}
{\tt 0}
\end{center}
You will next be asked for the numbers of the tape files to be copied to disk.
These must be entered with one file per line, or a range of files
per line.  Input is terminated by entering 0.  For example to read
files 4, 8, 12 through 16, and 20, type: \\
\begin{tabbing}
You will next be asked for the num\=   \kill
\> {\tt 4} \\
\> {\tt 8} \\
\> {\tt 12-16} \\
\> {\tt 20} \\
\> {\tt 0} 
\end{tabbing}
You will next be asked how you want the SDAS internal FITS data sets named.  
They will be placed in your present default directory in data sets $<$name$>$.HHH,
for the FITS header, and $<$name$>$.HHD, for the data.  Three methods of
choosing $<$name$>$ are available.  You will be asked if you want the names
taken from a keyword in the FITS header of the file.  If so, enter
the name of the keyword, otherwise just hit carriage return.
If you did not specify a keyword name then you will be asked for
a tape name.
If you give a tape name, (e.g.\ JUPITER),
then the disk data set names will be JUPITER1 (file 1), JUPITER2 (file 2), etc.
If you do not enter a tape name (i.e. just hit $<$return$>$), you will be asked 
to enter a data set name for every tape file specified.

Once the disk data set names have been specified, the program processes the
tape.  No further input is needed.

In the event any tape files also have FITS extension files, these
will be placed on disk with names $<$name$>$\_1, $<$name$>$\_2, \ldots where
$<$name$>$ is the name of the file.  Each extension file will have
a .HHH and .HHD disk file.

The SDAS internal FITS disk data sets can be read using the IDL commands SXOPEN,
and SXREAD.
 
The SDAS internal FITS data sets are NOT converted to floating point values, in
order to permit working with byte, half-word, or 32 bit integer data.
To rescale integer data to floating point, execute the following IDL
statements: \\
\\
{\tt SXOPEN,1,'filename',HEADER} \\
{\tt FDATA = SXREAD(1)*SXPAR(HEADER,'BSCALE')+SXPAR(HEADER,'BZERO')} \\
\\
{\bf FITSWRT - IDL FITS tape write routine.} \\
\\
To write internal SDAS formatted FITS data sets from disk to tape, use the
routine FITSWRT.  FITSWRT does not re-position a tape before writing to it,
so the tape should already be positioned using the REWIND or SKIPF comands.
If you want to append new FITS files to an existing FITS tape then first 
use the
procedure TINIT to position the tape between the final double EOF.
To begin execution of FITSWRT, while in IDL, type:
\begin{center}
{\tt FITSWRT}
\end{center}
You will then be asked for the tape unit number to write to.  For example
for a tape mounted on a tape unit assigned to MT1:, type: 
\begin{center}
{\tt 1} 
\end{center}
You will then be asked for the tape blocking factor,  enter 1
for a unblocked (2880 byte record) FITS tape, otherwise enter
the number of 2880 byte blocks per tape record.

You now have the option of having the file name placed in a
keyword of the header written to tape.  You may enter the name of
the keyword or just hit carriage return if you don't want the
file name written.

Next you will be asked if FITS extension files are to be search for
automatically. If you answer yes then for every fits file name
specified the program will search for all files name\_$\ast$.hhh and
write them to tape as extensions to name.

You will next be asked if the data sets to be written have names in the
form $<$name$><$number$>$, (eg. JUPITER1, JUPITER2, \ldots).  If you answer
yes then you will be asked to enter a value for $<$name$>$.  For the previous
example enter:
\begin{center}
{\tt JUPITER}
\end{center}
You will then be asked to enter the file $<$number$>$ for each data set to
be copied to tape.  For example to copy JUPITER3,
JUPITER5, JUPITER6, JUPITER7, JUPITER8, AND JUPITER2 to tape, type: \\
\begin{tabbing}
You will next be asked for the num\=   \kill
\> {\tt 3} \\
\> {\tt 5-8} \\
\> {\tt 2} \\
\> {\tt 0} 
\end{tabbing}
Notice input was again terminated by typing 0.

If the data set names are not in the above format you will be asked
to enter each data set name, one per line.  
Once you have entered all names, enter a null string (blank line)
by hitting $<$return$>$.
If you did not specify automatic searching for FITS extension
files you may enter them with the file names by typing:
\exone{name/ext.name1/ext.name2/ \ldots}
For example if the FITS file name was JUPITER and you had
a FITS table, juptab, associcated with it enter: 
\begin{center}
{\tt jupiter/juptab} 
\end{center}
Processing will then proceed until all files are copied.
The order on tape will be the same as the order specified in the list of
disk files.

FITSWRT can be used for byte, 16 bit integer, 32 bit integer or 32 bit
floating point data files.  Floating point data files are not converted
to scaled integers when writing to tape, and therefore may not be transported
to all computer systems.
\end{document}
