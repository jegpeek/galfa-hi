\documentclass[psfig,preprint]{aastex}
                                                                                
\begin{document}
                                                                                

\title{GALFA SPECTROMETER (GALSPECT): SETUP, OPERATION, BASICS}

\author{Carl Heiles (9/5/04), updated by Yvonne Tang (11/1/04), Jeff
Mock (11/6/04), Sne\v{z}ana (05/22/05)}
                                                                                

	This material is from Jeff Mock, the person that designed the 
spectrometer.  This document is a distillation of the full scoop, 
which is on the webpage {\it seti.berkeley.edu/galfa} . The distillation 
contains the following: 

\tableofcontents


\section{ STARTING GALSPECT} \label{start}

	The control program for GALSPECT is called {\it gdiag} . To run
GALSPECT for conventional Galactic HI observations, perform the following
steps: \begin{enumerate}

	\item {\bf Obtain the encryption file $galfa\_key$.} 
If this is the first
time you log into GALSPECT, you need to copy the file $galfa\_key$ 
from Jeff Mock's directory to your home directory. 
To do so, while working in your home directory, type the command

\vspace{.1in}
{\centerline{\tt cp ~jmock/galfa\_key .}}

	\item {\bf Log into the GALSPECT computer.} You should have
already copied the file $galfa\_key$ to your home directory
and made it read-only by the user ({\it chmod 0400 galfa$\_$key}). Then
the appropriate command is

\vspace{.1in}
{\centerline{\tt ssh -i galfa\_key galfa$@$galfa1 }}

	\item {\bf Check that no one else is already using GALSPECT},
i.e.\ that no one else is running {\it gdiag}

\vspace{.1in}
{\centerline{\tt  ps}}

\noindent This lists the programs running on the {\it galfa1}
computer. Look for {\it gdiag} . If it is already running, {\it STOP
HERE!} 

	\item {\bf Check that the data disk is mounted and has free space.}

\vspace{.1in}
{\centerline{\tt  df /dump  }}

	\item {\bf Run the basic operational check of GALSPECT.}

\vspace{.1in}
{\centerline{\tt  /var/diag}}

	Let this run for 30 sec or so.  If there are error messages,
reboot GALSPECT (see \S 3.1).  If you still get error
messages, repeat until it works or until you or the equipment 
die of exhaustion. 

	\item {\bf Set the DAC levels to 10 units rms.}

\vspace{.1in}
{\centerline{\tt  /var/levels}}


	Setup the telescope and move it near the starting source 
before setting levels.  The 1st IF and IF routing should be 
configured to provide a valid signal to galspect before setting
levels.

	Let this run until it finishes.  It prints out its current
action on the screen and, at the end, a summary.  In the summary, the
rms should be around 10, as you requested, and the DAC values should be
around 90.  Sometimes a large RFI pulse might interfere with this operation. 
If the levels do not look right, repeat this process.


{\it Note 1:} The gains increase with decreasing DAC number in a
{\it highly nonlinear way}.  From DAC=0 to $\sim 90$ the gain decreases
by $\sim 4$ db; from DAC$\sim 90$ to 255 the gain decreases by $\sim 36$
db.  Thus, low DAC values are very sensitive to signal levels.  High DAC
numbers should be very rare. A low DAC number is not a problem as long
as the rms is acceptable. 

{\it Note 2:} Setting levels depends on lo2! {\it /var/levels} stands for 
{\it gdiag -newdac=10 -lo2=256.26}.
If your lo2 is different from 256.26 MHz, meaning your central frequency
is different from 1420.40 MHz then you will have to modify this script.
For example, if your central frequency is 1385.0 MHz (used for E-ALFA
observations) then you need to use:
{\it gdiag -newdac=10 -lo2=195.845} .

	\item {\bf Decide how many one-second dumps per FITS file you want.}
The example below assumes 600 ({\it --sdiv=600}), meaning that each FITS 
file lasts 10 minutes.

	\item {\bf Decide on a project name.} This should normally be the
observing proposal number, e.g.\ A1943. 

	\item {\bf Create a shell script file with observation parameters.} 
The file is usually the observing proposal number, it should be 
placed in /var, make executable ({\it chmod 775 /var/a1943})  and contains a 
galspect command that looks something like this:

\vspace{.1in}


{\centerline{\tt  \#!/bin/sh }}
{\centerline{\tt  gdiag -galfa -sdiv=600 -scram -lo2=256.25 -proj=A1943 -vnc }}

	\item {\bf Start the observation.} 

\vspace{.1in}
{\centerline{\tt  /var/A1943}}

\noindent The {\it --time} option allows you to decide the number of seconds
you want to run the program.  If you want to run forever, don't include the {\it --time}
option. 

	The {\it --scram} option allows GALSPECT to listen to the network directly
with the LO values.  It replaces the obsolete option {\it --offset}.

	The {\it --lo2=256.25} option sets the frequency of the LO2
frequency synthesizer located in the galspect rack to 2*256.25 MHz.
The synthesizer is set to twice the LO2 frequency to account for the 
way the analog mixers work in galspect.

	The {\it --vnc} allows you to view the online graphical output
on your local terminal---and anyone else to view it on her terminal
(simultaneously).  And not only to view it, but also to change the
display and ({\it yes, it's true!}) the value of $digitalmix$.  So in
principle some random hacker---or your collaborator in Timbuktu---could
ruin your observation.  If you replace the {\it --vnc} by {\it --run},
then nobody can view the online output, and nobody---not even the local
keyboard on GALSPECT---can change things while you are running.  If you
invoke {\it gdiag} and specify neither {\it --vnc} nor {\it --run},
then GALSPECT's local display and keyboard both function. 

	If you are running from off-site you shouldn't use the display
because of network latency, so you should not use {\it --vnc}; and if
you want to make sure that local staff don't accidentally ruin your
data, use {\it --run}.

\vspace{.1in}


{\centerline{\tt  gdiag -run -scram -lo2=256.25 -sdiv=600 -proj=A1943 }}

	\item {\bf Open another xterm and invoke the online display with}

\vspace{.1in}
{\centerline{\tt  vncviewer galfa1}}

\noindent from LINUX machines, or 

\vspace{.1in}
{\centerline{\tt  /pkg/misc/bin/vncviewer galfa1}}

\noindent from SOLARIS machines. 

	When running $vncviewer$, you can change the display as
explained in its documentation (\S \ref{doc} below). Make sure that the RA, DEC, 
LO1 and LO2 values on the $vncviewer$ display match those on the main observation panel.

	It is advised that you do not run $vncviewer$ remotely.  
The more people running it, the more
network bandwidth it uses which might cause the lost of data.  

\item Finally, the files are located in 

%{\verb galfaserv.naic.edu:/export/galfa.startdate.project.sequence.fits }
{\verb /share/galfa/galfa.startdate.project.sequence.fits }

\end{enumerate}

\section{ STOPPING GALSPECT} \label{stop}

	To stop GALSPECT: \begin{itemize}

\item If you are running $vncviewer$, then stop GALSPECT by typing $q$
while the cursor is in the plotting window.

\item If you are not running $vncviewer$, then stop GALSPECT by typing
CTRL-c.

\end{itemize}

\section{ PROBLEMS WHEN RUNNING GALSPECT} \label{problems}

\subsection{Rebooting GALSPECT} \label{rebooting}

	When there is a problem and you need to reboot GALSPECT, 
you can first stop GALSPECT 
by typing $q$, then type in $ps$ to check the $pid$(process id) of $gdiag$.  
After that, type in $kill -9 pid$ to kill the process
and finally type $reboot$ to, well, reboot GALSPECT.

	When the above method does not work, you can reboot GALSPECT by powering down 
(turn the key) {\it for 1 minute}.  Power up and try again. 

	{\bf DO NOT REBOOT GALSPECT UNLESS ABSOLUTELY NECESSARY}.
Generally speaking, it is only absolutely necessary when 
the {\it /var/diag} test fails.

\subsection{Overflows}

	If the input gets too strong there are warning messages about
overflows. For narrowband spectra these messages contain the string
MLFS, in which each letter represents an internal digital operation,
followed by a four digit number, one digit for each operation. All
digital operations are done in integer arithmetic, and overflow means
just that. When overflow occurs, the number saturates at the maximum
value and there is no wraparound. The digits take on values from 0 to 3,
with larger numbers being increasing severity.

	For narrowband spectra, the operations are: \begin{enumerate}

	\item {\bf A} means the analog to digital converter.  The ADC
is overflowing (saturating) when this is set.

	\item {\bf M} means the digital Mixer. Dan says that saturation
is less serious than for the other processes. 

	\item {\bf L} means the digital Low Pass Filter. 

	\item {\bf F} means the Fourier transform computation.

	\item {\bf S} means the accumulator (Sum).

\end{enumerate}

	The meanings of the numbers are: \begin{enumerate}

\item {\bf 0} means almost perfect  (0-15 overflows during 1s intergration)

\item {\bf 1} means pretty good (16-255 overflows during 1s integration)

\item {\bf 2} means pretty bad  (256-4095 overflows during 1s integration)

\item {\bf 3} means horrible ($>=$ 4096 overflows during 1s integration)

\end{enumerate}

\noindent For wideband spectra there are only two relevant overflow
parameters, {\bf F} and {\bf S}. 

	Normally, when centered on Galactic HI, GALSPECT's wide (100
MHz) band covers about 1388 to 1488 MHz. Every 12 seconds the SJU radar,
centered at 1350 MHz, partially saturates the RF electronics and causes
saturation problems at the 1 level. These are not serious. It is
surprising if you don't see error messages every 12 seconds.  This can
change depending on time of day, azimuth angle, and possibly other
parameters.

	Normally, when the EGALFA people are observing, GALSPECT's
wideband is centered lower by about 30 MHz. The SJU radar comes directly
into the wide band and produces saturation problems at the 3 level,
which is very serious. But this {\it does not indicate a problem with
GALSPECT} and you should just keep observing. 

	GALSPECT has a few birdies. One is at the center channel, i.e.\
baseband DC. You'll need to interpolate over this, or center the lines
away from band center.

\subsection{Missing Records in the FITS output file.} GALSPECT writes
out about 2 GB per hour. If the output filesystem is being stressed by
another user, GALSPECT might complain that it has missed writing out
some records. This is serious: you are not recording data! Tell Arun;
this has happened before and he has tried to arrange that it will never
happen, so he is familiar with the problem and needs to know about it. 
If you want to investigate yourself, then stop GALSPECT and take a look
at the activity on your output file system. 

\subsection{What Time Is It?}

%\clearpage

	If there is a problem with time, check that the machine time on
galfa is NTP locked:

\begin{verbatim}
           ntpdc galfa1.naic.edu
           > peers

remote           local      st poll reach  delay   offset    disp
=======================================================================
*mosquito.naic.e 192.231.93.131   1  512  377 0.00104  0.000107 0.00780
=cuca.naic.edu   192.231.93.131  16 1024    0 0.00000  0.000000 0.00000
ntpdc>

\end{verbatim} 

The offset should be within a few ms; here, it is off 0.1 ms.

\subsection{Error setting frequency}

If you try to set levels and get:

\begin{verbatim}
LO2: Set failed, got back: ERROR setting freq
\end{verbatim}
this is a  serious problem and means that galfa1 can not talk properly to
wappserv to obtain observing information.
This error happened twice in May 2005 and both times required
restarting of the program gpibsock.
This should be done by Arun or Phil and not during an observing session.
The program that has to be restarted is called 
/share/wappsrc/bin/start\_gpib and it's on wappserv.


\section{ THE LO ARRANGEMENT FOR GALSPECT} \label{lo}

\subsection{ Frequencies}

	GALSPECT is a baseband spectrometer that samples complex inputs,
meaning that it separates negative and positive frequencies. Thus the
baseband center of each GALSPECT spectrum is DC. The IF is mixed with
the second LO, called $LO2$. The wideband baseband center frequency of
DC corresponds, at IF, to the frequency of $LO2$. The narrowband baseband
center frequency of DC corresponds, at IF, to the frequency of $LO2 -
digitalmix$, where $digitalmix$ is digitally generated within GALSPECT. 

\begin{figure}[!h]
\begin{center}
\includegraphics[width=4in]{galspect1fig.ps}
\end{center}

\caption{IF and RF frequencies for GALSPECT. $LO1 = 1695.4$ MHz,
$LO2=256.25$ MHz, $digitalmix=-18.75$ MHz.}
\end{figure}

	Suppose you are observing the HI line at 1420.400 MHz (for this
example, rounded from 1420.405752 MHz and no Doppler correction) and you
want it centered in the narrowband spectrum. To accomplish this, set the
first LO to

$$ LO1 = 1695.400 \ {\rm MHz} =1420.400+275.000 \ {\rm MHz} $$

\noindent This makes the IF line frequency 275.000 MHz. The IF center of
the wideband (${\rm width}=100$ MHz) GALSPECT spectrum is at the
frequency of $LO2$, normally set to 

$$ LO2 = 256.250 \ {\rm MHz} $$

\noindent and the IF center of the narrowband (${\rm width}={100 \over 14} =
7.142857$\dots \  MHz) is at $ LO2 - digitalmix$. Normally, $digitalmix =
-18.75$ MHz, so the narrowband IF center frequency is normally

$$ LO2 - digitalmix = 256.25 - (- 18.75) = 275.000 \ {\rm MHz} $$

\noindent At RF, the wideband spectrum is centered at 1439.150 MHz and
the narrowband one at 1420.400 MHz. 

	These frequencies are set as follows: \begin{enumerate}

	\item The $LO1$ frequency is set by the observing software.

	\item The $LO2$ frequency is set by the -lo2 commandline option. The
synthesizer is actually set to twice the LO2 frequency to account for 
the way the mixers operate. You just need to specify -lo2 in {\it
gdiag} and this will also set the synthesizer frequency
correctly. Also, remember to specify -lo2 option for setting levels
with {\it gdiag -newdac=10 -lo2=256.26}.

	\item The offset between centers of the wideband and narrowband
spectra, --18.75 MHz in this example, is the quantity $digitalmix$. You
can set it two ways, one {\it gdiag --mix=nn} and the other with the {\it
w} option in {\it vncviewer}. There are 32 possible values, spaced by
$100 \over 32$ MHz. For $nn < 16$ $digitalmix$ is negative, and for $nn
\ge 16$  $digitalmix$ is positive. The default value is -18.75 MHz.

\end{enumerate}

	These relationships are illustrated in the Figure.


\subsection{Channels}

	Both GALSPECT's wideband and narrowband spectra have RF
frequency increasing with channel number.  The wideband spectra has 512
channels and the center channel is number 256 (counting from zero). 
This wideband channel has a big DC spike. 

	Each narrowband spectrum has 8192 channels in the Fourier
transform computation.  When writing out to a file, 513 are removed
and replaced by other numbers (512 are the wideband
spectrum; the 512th is a flag). This leaves 7679 channels in the
narrowband spectrum. The center channel is number 3839 (counting from
zero), and again RF frequency increases with channel number.


\section{ OPTIONS FOR {\it gdiag} AND {\it vncviewer}} \label{doc}

	You can get this list by invoking {\it gdiag} with no options.

\begin{verbatim}
Usage: gdiag [options]

Main operating modes
    -adc       Print out buffer of ADC samples as text
    -rfft      Plot real FFT of ADC channel samples
    -cfft      Plot complex FFT of ADC channel samples
    -scope     Plot oscilloscope view of ADC samples
    -patt      Pattern test for data acquisition
    -dump      Print galfa acquisition as text
    -galfa     Plot galfa data and write FITS file
    -run       Collect galfa data and write a FITS file
    -dac       Set DACs for input level of f dBM

Other options
    -vnc       Run as VNC server instead of console
    -avg=n     Average interval for histograms and FFTs
    -max       Add max-hold line FFT displays
    -input=n   Take input from channel n
    -ppdb=f    Pixels per dB for vertical scale
    -adcfreq=f Use f as ADC sample frequency (MHz)
    -nshift=n  Set upshift of narrowband PFB before acc
    -wshift=n  Set upshift of wideband PFB before acc
    -npfb=x    Set narrowband PFB downshift vector
    -wpfb=x    Set wideband PFB downshift vector
    -beam=n    Select beam for single beam operations
    -scram	   Listen to the network with the LOs
    -mix=0..31 Select mixer for narrowband
    -ta=f      Signal generator A frequency
    -tb=f      Signal generator B frequency
    -lpf=x     Use LPF output instead of ADC for time domain displays
    -ppsint    Beam 0 gets PPS from internal source
    -proj=s    Project portion of filename for FITS dump file
    -sdiv=n    Number of seconds per FITS file
    -time=n    Run --run for n seconds
    -level=f   RMS units for analog level setting

During graphical operation
    Press `q' to quit program
    Press `p' to create raw image file in /tmp
    Press `r' to toggle max-hold
    Press `a' to toggle through galfa display modes
    Press `0-6' to select beam in galfa display
    Press `c/v' to modify pixels per dB on log display
    Press `z/x' to change pre-accum shift in galfa display
    Press `,/.' to scroll through narrow band displays
    Press `</>' to scroll faster through narrow band displays
    Press `o' to swap drawing order for polarizations
    Press `m/n/b' to manually move marker
    Press `w' to change mix frequency for narrowband
    Press `K/L' to zoom in/out x-axis in narrowband displays
    Press `d/f' decrease/increase PFB downshift vector
    Press `h' to toggle linear/log vertical display

Be careful not to hit `w' accidentally because it changes the mixer frequency 
and can cause serious problems.

\end{verbatim}

\section{ CHANGING THE PICTURE ON THE VNCVIEWER DISPLAY } \label{picture}

	It is understood that astronomers can get really bored sitting in the control room for hours.
Changing the small picture on the lower right corner of the vncviwer display can provide some entertainment.
To do so, rename a JPEG file (use one with a black background for best results) to {\it egg.jpg} and
{\it scp} it to the {\it /var} directory.  102x82 is a nice size for the
mascot image.


\end{document}
		
