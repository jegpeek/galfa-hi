\documentstyle[12pt,epsf]{article}
\setlength{\textheight}{21.5cm}
\setlength{\textwidth}{7in}
\setlength{\topmargin}{-0.25in}
\setlength{\oddsidemargin}{-0.25in}
\setlength{\evensidemargin}{0mm}
\setlength{\parindent}{0em}
\setlength{\headsep}{10.0mm}

\def\baselinestretch{1.0}
\pagestyle{myheadings}
%\markright{\bf Stanimirovic \& Muller: Arecibo Observing
%Proposal, Oct 2003}
\def\kms{km~s$^{-1}$}

\begin{document}
\parskip 5pt
\begin{center}
  {\Large \bf Instructions for running a2222}

\vspace{0.5cm} {\large Josh Goldston for the GALFA team (July 27 2006)}
\end{center}
%\vspace{0.5cm}

In case of any problems or concerns please call Josh at 510 299 4427, day or night.
These are observations with ALFA and GALSPECT. The observing procedure
consists of running a calibration routine (called Smart Frequency
Switching) and then making a map with basket-weave scanning (the observing
routine is called Basketweave Scanning). These procedures are {\it very} similar to 
those done with a2011, a2032, a2050 and a2060 so if you were familiar with those, you
should be familiar with this! 

Before we start, MAKE SURE THE ALFA COVER IS OFF.\\
Also, MAKE SURE THE AZ IS SET TO 360, NOT 0.\\
Here are some basic steps:

\begin{enumerate}

\item {\bf Starting CIMA and pointing the telescope}

\begin{enumerate}
\item On {\it observer2} login as dtusr. 
\item Open an xterm and type {\bf cima}. If it asks which version,
choose {\bf``normal''} . 
\item On the center window entitled ``Welcome to CIMA''
enter your name, and project number = a2222. Then under ``Select Observing Mode'' select ``Line''.
\item On the window ``Available Receivers''
click on ``ALFA'', then click on ``Disable Quick Tsys'', then click on
``Select Receiver Now''.
This will start rotating the turret to ALFA.

\item Re-start WAPPs with:
On the ``CIMA Observer's Interface'' window click on ``Utilities''. This will open a new window; click on ``Restart WAPPs''. You can now dismiss the last window.

\item From ``CIMA Observer's Interface'' window select ``Command File
Observing''. 

\item A new window, ``Command File Observing'' will pop up.
Click on ``Load'' to go and browse for a file you want to
run. Click on file ``acwf\_day\_00\_pt1'', click on ``Open''.
Then back on  the ``Command File Observing'' window
click on ``Run''. 
This will load the IFLO setup file and start driving the telescope to
the desired source.


\end{enumerate}


\item {\bf Starting GALSPECT}

\begin{enumerate}
\item While the telescope is driving to the desired position, 
Login to {\it dataview} as user `guest' (password is naic305m). 

\item Open an xterm and login to galfa1 computer by typing:

{\bf
[guest$@$dataview guest]\$  ssh -i galfa\_key galfa$@$galfa1 
}

The prompt {\bf \#} appears.

\item
In the same window type:

{\bf
\# /var/diag
}

Let it run -  you will see lots of messages, as GALSPECT is warming up. 

\item
Once you get on the source in the same window type:

{\bf
\# /var/levels
}

If this gives the message {\bf LO2: Set failed, got back: ERROR
setting freq} follow the procedure given in the footnote\footnote{
1. Open a new xterm on {\it dataview}.\\
2. login to wappserv as user wapp (password=wappme) by typing {\bf ssh
wapp$@$wappserv}. It will ask for password, type {\bf wappme}.\\
3. Type {\bf /home/cima/Wapp/Bin/Progs/Start/st art\_gpib}.\\
4. Return to the previous step and type again {\bf /var/levels} in the
{\it galfa1} window.}.
This will finish quickly and give you a short summary. 
Check RMS values, they should all be around 10. 
If they are not close to 10 call Josh.

\item Now, in the same window type:

{\bf
\# /var/a2222
}

This will mean you are starting to write data files with the GALSPECT. 
You should start to see numbers tick by: 5s, 10s, 15s etc. You may also see some error
messages - this is OK. The only serious problem is if this window does not print 
new things of some kind every 5 seconds or so. If the window is completely frozen (no new 
text) type CTRL-C in the window and type:

{\bf
\# /var/a2222novnc
}

This mode should again give you numbers that tick by. If you still do not see these numbers,
call Josh. NOTE: If you end up using this mode, the OPTIONAL STEP below,
that allows you to view what GALSPECT is doing, will not work. This is OK.

\end{enumerate}

\item {\bf Running the calibration and observations}

\begin{enumerate}
\item Return to cima on {\it observer2}.
and to the ``Command File Observing'' window. 

\item 
Click on ``Load'' to go and browse for a file you want to
run. Click on file ``acwf\_day\_00\_pt2'', click on ``Open''.
Then back on  the ``Command File Observing'' window
click on ``Run''. 

\item
This will start a series of steps, it will run a calibration routine first,
then a basket-weave scan. 
Please WATCH the ``AO Observer Display'' for a few minutes! It should
show updated messages every few seconds. If you notice that 
observing is hanging (you don't see updated messages every few
seconds) please call Josh. If the telescope is `waiting' at to begin
the basketweave scan for some number of seconds, that is OK.

\item
Please take note of what object the telescope the tracking for the majority of the night.
The name should be something like ``acwf\_16\_01''. Please send this name, along
with any abnormalities or difficulties you experience during the night in an e-mail to 
Josh at golston@astro.berkeley.edu.


\item At any time during observing you can open 
``Quick Look Data Display'' on {\it dataview} to make sure the spectra
   are being updated. \\

\item OPTIONAL STEP: You can also view data using GALSPECT's
display. 

\begin{enumerate}
\item Open a new xterm on dataview and type:\\
{\bf [guest$@$dataview guest]\$ vncviewer galfa1} \\

\item
This brings up a {\it plot window} entitled ``TightVNC: Pixmap
framebuffer''. With the cursor on this window type ``h'', this will
blow up the plot and make it easier to inspect spectra.

\item 
G-ALFA folder in the control room explains how to change different display
options
\end{enumerate}

\end{enumerate}

\item {\bf Stopping at the end of the run}

\begin{enumerate}
\item  On {\it observer2}, on the ``AO Observer Display'' click on ``Abort
Observation''. Wait a moment until everything stops. If it doesn't
stop keep typing ``Abort Observation'' until it does.

\item On {\it dataview's} galfa1 window 
(the one which prints numbers all the time) 
press Control-C, and exit from this window. 
If you forget to do this, GALSPECT will take data
all night!

\item On {\it observer2}, on ``Observer's Interface'' click on 
``Exit CIMA'' to exit from cima. Close all windows. 
\end{enumerate}
\end{enumerate}

{\bf Thank you very much!}\\

Josh et al

\end{document}





