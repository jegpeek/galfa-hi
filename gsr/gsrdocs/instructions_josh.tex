\documentstyle[12pt,epsf]{article}
\setlength{\textheight}{21.5cm}
\setlength{\textwidth}{7in}
\setlength{\topmargin}{-0.25in}
\setlength{\oddsidemargin}{-0.25in}
\setlength{\evensidemargin}{0mm}
\setlength{\parindent}{0em}
\setlength{\headsep}{10.0mm}

\def\baselinestretch{1.0}
\pagestyle{myheadings}
\def\kms{km~s$^{-1}$}

\begin{document}
\parskip 5pt
\begin{center}
  {\Large \bf Instructions for running a2011}

\vspace{0.5cm} {\large Carl Heiles, Josh Goldston 
(31 May 2005)\footnote{Adapted from
``Instructions for running a2032'' by Snezana Stanimirovic (May 22 2005).}}
\end{center}

{\bf If you get started more than eight minutes late, please call Josh
Goldston at 510 299 4427.}

In case of any problems or concerns please call Josh Goldston at 510 299
4427.  These are observations with ALFA and GALSPECT.  The observing
procedure consists of running a calibration routine (called Smart
Frequency Switching) and then making a map with basketweave scanning
(the observing routine is called Basketweave Scanning). 

%\Huge

\noindent {\bf ********************************************************************* 

MAKE SURE THE ALFA COVER IS OFF!!!

\noindent *********************************************************************}

%\normal

There are four major parts to the process:
\begin{enumerate}

\item {\bf Starting CIMA and pointing the telescope}

\item {\bf Starting GALSPECT}

\item {\bf Running the calibration and observation}

\item {\bf Stopping at the end of the run}
\end{enumerate}


Here we go:

\begin{enumerate}

\item {\bf Starting CIMA and pointing the telescope} \begin{enumerate}

\item On {\it observer2} login as dtusr.

\item Open an xterm and type {\bf cima}. If it asks which version, choose ``test''.

\item On the center window entitled ``Welcome to CIMA'' enter your name
and project number = a2011 (note ``a`` is lower case).  Then under
``Select observing mode'' select ``Line''.

\item A new window entitled ``Available Receivers'' pops up.  Click
``ALFA'', click ``Disable Quick Tsys'', and then ``Select Receiver
Now''.  This will start rotating the turret to ALFA.  Finally, click on
``DISMISS'' to get rid of this window.

\item On the ``Observers Interface'' window (upper left of screen) click
on ``Command File Observing''.  

\item A new window opens up near the upper right labelled ``Command
File Observing''.  Click ``Command file''; a new window opens up
entitled ``Select Command File''.  

\item In this new ``Select Command File'' window you see a set of files
displayed.  We have two files, a0part1 for pointing the telescope and
setting things; and a0part2 for doing the calibration routines and the
actual observations themselves.  Select {\bf a0part1} and then click on
``Open''. 

\item Return to the ``Command File Observing'' window near the upper
right of the screen. Click ``Start Command Line Observation''.

\item At this point look at the ``AO Observer Display'' window near the
bottom right of the screen.  At the bottom left of this window the word
``Observing'' should be highlighted in green.  Also, this window prints
out a line every time it does something.  It should progress in a normal
fashion.  

The telescope should start driving to the source. 

If there are any {\it red} messages, then there is a {\it problem}.  If so, call Josh
Goldston at 510 299 4427. 

\end{enumerate}

\item {\bf Starting GALSPECT} \begin{enumerate}

\item While the telescope is driving to the desired position, login to
{\it dataview} as user ``guest'' (password is ``naic305m''). 

\item On {\it dataview}, open an xterm; the prompt \verb%[guest@dataview
guest]$% appears.  In this window, login to the {\it galfa1} computer by
typing:

{\bf ssh -i galfa\_key galfa$@$galfa1 }

The prompt \verb$#$ appears, showing that this window is now a {\it
galfa1} window. 

\item In this same {\it galfa1} window type:
{\bf ps $|$ grep gdiag }

If it prints out anything, then type {\bf kill PID }, where PID is the first number.

\item In the same {\it galfa1} window type:

{\bf /var/diag }

Let it run for some time, like 30 sec, you will see lots of messages,
GALSPECT is warming up. Stop this by typing Control-C.

\item Wait until the telescope gets on the source.  Then in the same
{\it galfa1} window type:

{\bf /var/levels }

If this gives the message {\bf LO2: Set failed, got back: ERROR setting freq},
follow the procedure in the footnote\footnote{1. Open up a new xterm on
{\it dataview}. \\
2. login to wappserv as user wapp (password=wappme) by typing {\bf ssh
wapp@wappserv}. It will ask for the password: type {\bf wappme} .\\
3. Type {\bf source /share/wappsrc/bin/start\_gpib} . \\
4. Then return to the previous step where you type
{\bf /var/levels} in the {\it galfa1} window.}.  Usually it will 
write out a whole bunch of numbers. 
The important ones to look at are the summary at the end. This consists
of 28 RMS values, 4 for each beam. 
A normal screen isn't big enough to display them all, so you have to
scroll up to see them all.  These RMS values should all be near 10, say
between 7 and 14 or so.  If they are not, try it again.  And again.  And
again.  If the RMS values keep being wrong, then call Josh Goldston at
510 299 4427. 

\item FINALLY! We are set to observe. In the {\it galfa1} window type:

{\bf /var/a2011}

Again, note that the ``a'' is lowercase. Typing this starts GALSPECT and
it will begin to write data files. Also, every 5 seconds it will write
the number of seconds elapsed on the {\it galfa1} window.

\end{enumerate}
\item {\bf Running the calibration and observation} \begin{enumerate}

\item Return to cima on {\it observer2}.  In the upper right window
labeled ``Command File Observing'' again click ``Command file''; a new
window opens up entitled ``Select Command File''. 

\item This time select select {\bf a0part2} and then click on
``Open''. 

\item Return to the ``Command File Observing'' window near the upper
right of the screen. It has expanded horizontally and you'll have to
drag it to the left to see it all. Click ``Start Command Line Observation''.

\item The first action is a calibration, which takes a few minutes. This
must finish before the telescope starts moving, so we try to leave
plenty of time.  As a result the telescope might not start moving for
several minutes; this is not a problem. 

\item {Optional step during observing: HAPPY BIRTHDAY!!} 

	At any time during observing you can view data using GALSPECT's
display: \begin{enumerate}

\item Open a new xterm on {\it dataview} and type:

{\bf vncviewer galfa1} 

This brings up a {\it plot window} entitled ``TightVNC: Pixmap framebuffer''.

\item Move the cursor onto this {\it plot window}, type ``a'' three times.

\item With the cursor to this plot window and type ``h'' once.

\item With the cursor on this plot window, type ``L'' in UPPERCASE.


\item At this point you should see seven plots, one for each of ALFA's
beams.  Each plot has two graphs, one yellow and one green.  They should
look similar, like a flat-topped {\it lemon-lime birthday cake} viewed
from the side (yum-yum!).  In the very center should be a narrow 21-cm
line with height about the same as that of the cake; regard the line as
a birthday candle in the very center of the cake.  (During the
calibration the candle moves around and sometimes disappears.)

	If you don't like the birthday cake, or if you are getting
bored, you can change the display by typing various combinations of
``a'', ``h'', and ``L''. {\it But whatever you do, DON'T TYPE w OR x}
because this will make big changes in the GALSPECT configuration.

	These plots should continually update once per second. If this
doesn't happen, then call Josh Goldston at 510 299 4427. 

\item The G-ALFA notebook in the control room has more details on GALSPECT
operation. 

\end{enumerate}
\end{enumerate}


---------------------------------  much time passes observing  ----------------------------------

\item {\bf Stopping at the end of the run} \begin{enumerate}

\item On {\it observer2}, on the ``AO Observer Display'' click ``Abort
Observation''. Wait a moment until everything stops. If it doesn't stop,
keep typing ``Abort observation'' until it does.

\item On {\it dataview}'s {\it galfa1} window (the
one which prints numbers all the time) press Control-C. This should stop
GALSPECT. If it doesn't, keep hitting Control-C.

\item On {\it observer2}, on the ``Observers Interface'' window, click ``Exit
Normally'' to exit from CIMA. 

\end{enumerate}


\end{enumerate}

{\bf Thank you very much!}

Carl and Josh

\end{document}





