\documentstyle[12pt,epsf]{article}
\setlength{\textheight}{21.5cm}
\setlength{\textwidth}{7in}
\setlength{\topmargin}{-0.25in}
\setlength{\oddsidemargin}{-0.25in}
\setlength{\evensidemargin}{0mm}
\setlength{\parindent}{0em}
\setlength{\headsep}{10.0mm}

\def\baselinestretch{1.0}
\pagestyle{myheadings}
\def\kms{km~s$^{-1}$}

\begin{document}
\parskip 5pt
\begin{center}
  {\Large \bf Instructions for running a2011}

\vspace{0.5cm} {\large Carl Heiles, Josh Goldston 
(27 May 2005)\footnote{Adapted from
``Instructions for running a2032'' by Snezana Stanimirovic (May 22 2005).}}
\end{center}

{\bf If you get started more than eight minutes late, please call Josh
Goldston at 510 299 4427.}

In case of any problems or concerns please call Josh Goldston at 510 299
4427.  These are observations with ALFA and GALSPECT.  The observing
procedure consists of running a calibration routine (called Smart
Frequency Switching) and then making a map with basketweave scanning
(the observing routine is called Basketweave Scanning). 

Here we go:

\begin{enumerate}

\item On {\it observer2} login as dtusr.

\item Open an xterm and type {\bf cima}.

\item On the center window entitled ``Welcome to CIMA'' enter your name
and project number = A2011 (note ``A`` is CAPITALIZED).  Then under
``Select observing mode'' select ``Line''.

\item A new window entitled ``Available Receivers'' pops up.  Click
``ALFA'', click ``Disable Quick Tsys'', and then ``Select Receiver
Now''.  This will start rotating the turret to ALFA.  Finally, click on
``DISMISS'' to get rid of this window.

\item On the ``Observers Interface'' window (upper left of screen) click
on ``Command File Observing''.  

\item A new window opens up near the upper right labelled ``Command
File Observing''.  Click ``Command file''; a new window opens up
entitled ``Select Command File''.  

\item In this new ``Select Command File'' window you see a set of files
displayed.  Our observing files have names like ``\verb$LWS_day_5$''.  We want
to sequentially run through the days.  Every time a file is done
successfully, check it off on the attached list.  Select the next file
by clicking on it.  For example, if all files up to ``\verb$LWS_day_4$'' have
been observed already, select ``\verb$LWS_day_5$''. 

Then click on ``Open''.

\item Return to the ``Command File Observing'' window near the upper
right of the screen. Click ``Start Command Line Observation''.

\item At this point look at the ``AO Observer Display'' window near the
bottom right of the screen.  At the bottom left of this window the word
``Observing'' shold be highlighted in green.  Also, this window prints
out a line every time it does something.  It should progress in a normal
fashion.  

The telescope should start driving to the source. 

	If the yellow word ``IDLE'' appears instead of the green
``Observing'' at the bottom left of this window, or if there are any
{\it red} messages, then there is a {\it problem}.  If so, call Josh
Goldston at 510 299 4427. 

\item While the telescope is driving to the desired position, login to
{\it dataview} as user ``guest'' (password is ``naic305m''). 

\item On {\it dataview}, open an xterm and login to the {\it galfa1}
computer by typing:

{\bf
[guest$@$dataview guest]\$  ssh -i galfa\_key galfa$@$galfa1 
}

This links this window to the {\it galfa1} computer.

\item In this same {\it galfa1} window type:
{\bf \# ps }

This makes a list of processes running on the {\it galfa1} computer.  If
the word {\it gdiag} exists in the right hand column then either
somebody forgot to turn off the spectrometer or someone else is using
it. You need to kill this process. Do so by typing

{\bf \# kill PID }, where PID is the number in the first column.

\item In the same {\it galfa1} window type:

{\bf \# /var/diag }

Let it run for some time, like 30 sec, you will see lots of messages,
GALSPECT is warming up. Stop this by typing Control-C.

\item Wait until the telescope gets on the source.  Then in the same
{\it galfa1} window type:

{\bf \# /var/levels }

This will finish quickly and give you a short summary.  Check RMS
values, they should all be around 10.  If they are not close to 10, try
it again.  And again.  And again.  If the RMS values keep being wrong,
then call Josh Goldston at 510 299 4427. 

	\item When you type the above command there is a possibility that
you will see the message

{\bf LO2: Set failed, got back: ERROR setting freq}

If you see this message, then do the following: \begin{enumerate}

	\item Open up a new xterm on {\it dataview}.

	\item login to wappserv as user wapp (password=wappme) by typing

{\bf ssh wapp@wappserv}

It will ask for the password: type {\bf wappme} .

	\item Type

\verb$ source /share/wappsrc/bin/start_gpib$

	\item Then return to the previous step where you type
\verb$/var/levels$ in the {\it galfa1} window.

\end{enumerate}

                                                                        
\item FINALLY! We are set to observe! In the {\it galfa1} window type:

{\bf \# /var/A2011 }

Again, note that ``A'' is capitalized.  Typing this starts GALSPECT and
it will beigin to write data files with the GALSPECT. It will begin to
write bunches of numbers on the {\it galfa1} window.

\item On {\it dataview} select ``Quick Look Data Display'' and make sure the spectra
   are being updated. 

\item The first action is a calibration, which takes a few minutes. This
must finish before the telescope starts moving, so we try to leave
plenty of time.  As a result the telescope might not start moving for
several minutes; this is not a problem. 

---------------------------------------------------------------------------


\item  To stop at the end of the run: \begin{enumerate}

\item On {\it observer2}, on the ``AO Observer Display'' click ``Abort
Observation''. Wait a moment until everything stops. If it doesn't stop,
keep typing ``Abort observation'' until it does.

\item On {\it dataview}'s {\it galfa1} window (the
one which prints numbers all the time) press Control-C. This should stop
GALSPECT. If it doesn't, keep hitting Control-C.

\item On {\it observer2}, on the ``Observers Interface'' window, click ``Exit
Normally'' to exit from CIMA. 

\end{enumerate}


\item OPTIONAL STEP DURING OBSERVING: HAPPY BIRTHDAY!!: 

	You can view data using GALSPECT's display: \begin{enumerate}

\item Open a new xterm on {\it dataview} and type:

{\bf [guest$@$dataview guest]\$ vncviewer galfa1} 

This brings up a plot entitled ``TightVNC: Pixmap framebuffer''.

\item With the cursor on this plot window, type ``a'' three times.

\item Move the cursor to this {\it plot window} and type ``h'' once.

\item With the cursor on this plot window, type ``L''.


\item At this point you should see seven plots, one for each of ALFA's
beams.  Each plot has two graphs, one yellow and one green.  They should
look similar, like a flat-topped choclate birthday cake viewed from the
side (yum-yum!).  In the very center should be a narrow 21-cm line with
height about the same as that of the cake; regard the line as a birthday
candle in the very center of the cake. 

	These plots should continually update once per second. If this
doesn't happen, then call Josh Goldston at 510 299 4427. 

\item The G-ALFA notebook in the control room has more details on GALSPECT
operation. 

\end{enumerate}\
\end{enumerate}\

{\bf Thank you very much!}

Carl and Josh

\end{document}





